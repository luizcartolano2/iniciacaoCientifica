\section{TRABALHOS RELACIONADOS} \label{sec:trabalhos}
O deslocamento constante das pessoas, em busca de bens e serviços de qualidade, empregos e outras situações, vem acarretando cada vez mais problemas inerentes a mobilidade urbana, tais como, longos engarrafamentos, aumento da poluição e prejuízos financeiros, principalmente nas regiões metropolitanas, devido a grande concentração de veículos nessas áreas \cite{vianna2015busca}. A fim de minimizar tais problemas, os planejadores das cidades efetuam recorrentes melhorias na infraestrutura das vias, contudo, o alto custo e o longo tempo das obras são fatores que inviabilizam sanar esses problemas. Alternativamente, os Sistemas de Transportes Inteligentes (ITS na sigla em inglês) vêm sendo explorados para operar e prestar melhores serviços, e assim, garantir que a mobilidade urbana ocorra de forma mais eficiente, sem resultar em altos custos de implantação, tal prática já vem, inclusive, sendo usadas em alguns países, como mostra \cite{ref:port_its}.

O uso de dados providos de redes sociais (como o Twitter, Instagram, Foursquare, etc) para prever situações do mundo real tem se tornado cada vez mais comum nos dias de hoje. Afim de compreender a dinâmica das cidades, em \cite{silva2014you} , os autores exploram dados compartilhados no Foursquare para identificar limites culturais e preferências alimentares. Em trabalhos como o de Gomide \cite{gomide2011dengue}, no qual ele analisou como epidemias de dengue se refletem no uso do Twitter pelas pessoas das regiões afetadas pela doença. Ou ainda, Bollen \cite{bollen2011twitter}, que também usou do \textit{feed} do Twitter para analisar como o humor das pessoas influenciava no  valor da \textit{Dow Jones Industrial Average} (DJIA) ao longo do tempo. Já em \cite{sakaki2010earthquake} usuários do Twitter foram usados como sensores para identificar o acontecimento de terremotos.

No âmbito da mobilidade urbana, trabalhos como os citados também se popularizam. Trabalhos como \cite{ribeiro2014studying}, correlacionam mensagens compartilhadas por usuários do redes sociais com condições de tráfego para identificar padrões no comportamento das pessoas através de dados do Foursquare e do Instagram, as quais são correlacionadas com as condições de tráfego\footnote{fornecidas pelos mapas do Bing} e foi possível identificar uma maior atividade das pessoas nos períodos de pico de movimentação, nos mostrando uma relação entre a movimentação das pessoas e a atividade em redes socias. Similarmente, em \cite{NASSAR2017} vemos uma série de exemplos de aplicações nas quais aparelhos celulares transmitem, por meio de \emph{Near-Field Communication}(NFC) passageiros recebem, aproximando o aparelho celular do ônibus, informações cruciais à respeito do mesmo, como todos os horários e pontos nos quais aquela linha passou, entre outras. Auxiliando a passageiros que não possuem muito conhecimento sobre a rede da cidade, ou ainda aqueles que realizam muitas trocas de linhas.
