\section{MOTIVAÇÃO E OBJETIVOS}
Nos dias atuais, os sistemas computacionais estão evoluindo rumo à consolidação do paradigma da computação ubíqua \cite{weiser1991computer}, permitindo um mundo totalmente monitorado, através de sensores que vão desde dispositivos integrados ao ambiente físico até as redes sociais. Neste contexto, estima-se que por volta do ano de 2020 a quantidade de dados gerados por estes sensores atinjam a ordem dos 25 ZB (zettabytes) \cite{perera2014context}.

Na mesma proporção em que cresce a quantidade de dados gerados por sensores, cresce também a quantidade de veículos e o trânsito nas cidades do Brasil, como pode ser visto em \cite{ANTENOR2010}. Por conseguinte, cada vez mais problemas de saúde surgem na vida dos moradores dessas cidades, caracterizando um claro e gravíssimo problema nacional.

Para que esta nova tendência - \emph{da computação ubíqua} - torne-se realmente útil, e seja capaz mudar a vida das pessoas, faz-se necessário que novas aplicações sejam desenvolvidas tendo em mente a necessidade de se extrair um significado de todas estas diferentes fontes e tipos de dados, que seja relevante para a solução de problemas que afetem a vida das pessoas. Com base nesta necessidade, o objetivo deste trabalho é propor o desenvolvimento de uma estratégia para a geração de contexto \cite{abowd1999towards} através da fusão de diferentes fontes e tipos de dados sensoriados. Como prova de conceito para a viabilidade desta estratégia, será desenvolvida uma ferramenta para a realização de previsão do trânsito para o cenário de computação ubíqua.
