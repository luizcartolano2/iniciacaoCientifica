\section*{RESUMO}
O Twitter é, sem sombra de dúvidas, uma das plataformas que revolucionaram o mundo nos últimos anos, em especial, no âmbito de como as pessoas se comunicam. Nele, usuários falam sobre os mais diversos assuntos, desde seus problemas de saúde até problemas que encontram no seu dia a dia, como o trânsito, além de indicarem onde estão ou estiveram por meio de check-ins. Nesse paper, buscamos usar o Twitter como uma maneira de analisar o trânsito de grandes centros urbanos, para que, por meio de sistemas de transporte inteligente, possamos resolver o problema da mobilidade urbana no Brasil. O trabalho consistiu em dois momentos, em um primeiro período focou-se na coleta de dados do Twitter e do Google Maps. Enquanto que no segundo momento, buscou-se analisar a movimentação de pessoas usando a geo-localização dos tweets e, comparando com dados obtidos pelo Google Maps, prever o trânsito na cidade de Campinas.

\section{INTRODUÇÃO}
O Twitter está entre as redes sociais que modificaram a maneira como nos comunicamos. Somente no Brasil, existe, segundo \cite{ref:statista}, cerca de 18 milhões de usuários ativos no Brasil, e mais de 300 milhões no mundo. Estes são responsáveis por gerar cerca de 500 milhões de tweets (nome dado as mensagens postadas na rede) por dia, de acordo com \cite{ref:twitter_stats}. Nesses tweets, usuários falam sobre os mais diversos assuntos, desde seus problemas de saúde até problemas que encontram no seu dia a dia, como o trânsito, além de indicarem onde estão ou estiveram por meio de check-ins.

Defini-se por mobilidade urbana a condição criada para as pessoas poderem se locomover entre as diferentes zonas de uma cidade. Atualmente, os automóveis particulares e os meios de transportes públicos são os meios de mobilidade urbana mais utilizados. Como nos mostra Vianna  Young em \cite{vianna2015busca}, a mobilidade urbana é um dos grandes problemas enfrentados pela população brasileira atualmente, especialmente para os moradores de grandes centros urbanos. A baixa mobilidade dos grandes centros tem acarretado em uma série de problemas para a população. Alguns deles são o tempo perdido no trânsito, em São Paulo, por exemplo, a população em média 105,23 minutos por dia. Além disso, os problemas de saúde ocasionados que se tornam cada vez mais comuns, como: dores musculares, lombares e cervicais, problemas de circulação sanguínea, além do estresse e dos problemas respiratórios. Além disso, não podemos deixar de comentar sobre os problemas financeiros causados, o estado do Sudeste perde cerca de 3\% do seu PIB por causa do trânsito, como nos mostra \cite{vianna2015busca}.

Neste projeto buscamos um método alternativo para resolver o problema destacado. A partir de uma fusão de dados de sensoriamento participativo deseja-se encontrar uma relação entre o trânsito na cidade de Campinas-SP, e participação de usuários em redes sociais. 

Por meio de \emph{scripts} simples feitos em \emph{Python} e usando \emph{Interfaces de programação de aplicações - APIs}, disponibilizadas pelas próprias empresas(como a Google, Twitter, e.g), fizemos a parte inicial do projeto, que dizia respeito a coleta dos dados que serão utilizados no decorrer do trabalho. Depois de coletados os dados, foram feitas funções em \emph{Python} que fossem capaz de, em primeiro lugar, tratar os arquivos obtidos, fazendo com que eles estivessem em condições de serem processados. 

Em seguida , a partir dos dados processados do \emph{Twitter} criou-se \emph{heat maps} usando a \emph{geolocalização} dos tweets, de modo que foi possível identificar a movimentação dos usuários da rede social. Em paralelo, também tratamos dados oriundos do \emph{Google Maps}, a partir de um conjunto de rotas monitoradas, traçamos dois planos de uso para elas. Em primeiro lugar, montou-se uma página \emph{HTML} simples, que exibe informações detalhadas de cada rota obtida (tempo de duração, distância, horário, e.g). Além disso, no momento, trabalha-se na criação de um \emph{grafo}, no qual as ruas da cidade de Campinas serão suas arestas, e o tempo gasto para percorrer as rotas traçadas serão utlizados como \emph{"peso"} para elas, a fim de criar uma visualização mais dinâminca e rápida.

Por fim, pretendemos criar um \emph{serviço} no qual disponibilizaremos a solução criada para que todos os moradores da cidade possam utilizar.A implementação citada será explicada em mais detalhes na \autoref{sec:proposta}.